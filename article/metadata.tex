% DO NOT EDIT - automatically generated from metadata.yaml

\def \codeURL{https://github.com/CoraliePicoche/brownian_bug_fluid/tree/main/code}
\def \codeDOI{10.5281/zenodo.6546471}
\def \codeSWH{}
\def \dataURL{}
\def \dataDOI{}
\def \editorNAME{Pierre de Buyl}
\def \editorORCID{0000-0002-6640-6463}
\def \reviewerINAME{Francesco Turci}
\def \reviewerIORCID{0000-0002-0687-0715}
\def \reviewerIINAME{Rajesh Singh}
\def \reviewerIIORCID{0000-0003-0266-9691}
\def \dateRECEIVED{24 August 2021}
\def \dateACCEPTED{03 May 2022}
\def \datePUBLISHED{13 May 2022}
\def \articleTITLE{[Re] Reproductive pair correlations and the clustering of organisms}
\def \articleTYPE{Replication}
\def \articleDOMAIN{Ecology}
\def \articleBIBLIOGRAPHY{bibliography.bib}
\def \articleYEAR{2022}
\def \reviewURL{https://github.com/ReScience/submissions/issues/58}
\def \articleABSTRACT{In the present work, we replicate the results of Young et al. (2001) ``Reproductive pair correlations and the clustering of organisms'', an analysis of the formation of aggregates in an otherwise homogeneous environment mimicking marine small-scale hydrodynamics. Using an individual-based model of independent, random-walking particles (also called ``Brownian bugs''), they show that reproduction by fission in a turbulent and viscous flow leads to the formation of elongated clusters. Spatial patterns therefore depart from the usual, homogeneous solution of the advection-diffusion-reaction equation for a large population. Due to their size, phytoplankton organisms experience a mostly viscous environment in a laminar shear field, with random but homogeneous changes in directions due to turbulence. Reproduction and limited movement of daughter cells, which occur at the phytoplankton scale, interact with these hydrodynamics processes and can lead to aggregates.  In this context, a better understanding of the interactions between demography and small-scale hydrodynamics could provide further explanation for observed spatial distribution of phytoplankton species, and perhaps even their coexistence. This motivated us to revisit Young et al. (2001). In addition to replicating the numerical and mathematical results of Young et al. 2001, we also wished to present the mathematical derivations that were missing from the original paper, which should make this replication article more accessible to most readers, especially those without a fluid mechanics background.}
\def \replicationCITE{Young, W. R., Roberts, A. J., & Stuhne, G. (2001). Reproductive pair correlations and the clustering of organisms. Nature, 412(6844), 328-331}
\def \replicationBIB{young_reproductive_2001}
\def \replicationURL{https://www.researchgate.net/profile/William-Young-22/publication/11882165_Reproductive_pair_correlations_and_the_clustering_of_organisms/links/0c96052a9ecabaeec5000000/Reproductive-pair-correlations-and-the-clustering-of-organisms.pdf}
\def \replicationDOI{10.1038/35085561}
\def \contactNAME{Coralie Picoche}
\def \contactEMAIL{coralie.picoche@u-bordeaux.fr}
\def \articleKEYWORDS{phytoplankton, individual-based model, clustering, advection-diffusion-reaction, C++}
\def \journalNAME{ReScience C}
\def \journalVOLUME{8}
\def \journalISSUE{1}
\def \articleNUMBER{3}
\def \articleDOI{10.5281/zenodo.6546488}
\def \authorsFULL{Coralie Picoche, William R. Young and Frederic Barraquand}
\def \authorsABBRV{C. Picoche, W.R. Young and F. Barraquand}
\def \authorsSHORT{Picoche, Young and Barraquand}
\title{\articleTITLE}
\date{}
\author[1,\orcid{0000-0002-0867-2130}]{Coralie Picoche}
\author[2,\orcid{0000-0002-1842-3197}]{William R. Young}
\author[1,\orcid{0000-0002-4759-0269}]{Frederic Barraquand}
\affil[1]{Institute of Mathematics of Bordeaux, CNRS \& University of Bordeaux, Talence, France}
\affil[2]{Scripps Institution of Oceanography, University of California at San Diego, La Jolla, California, USA}
